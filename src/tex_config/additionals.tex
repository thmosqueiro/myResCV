%
% Zerando o tamanho da barra
%
\setlength{\headwidth}{0cm}


%
% Definindo as pastas onde as figuras estao
%
\graphicspath{{images/}, {images/future_works/}, {images/experiments/}, {images/logos/}, {images/our_results/}, {images/explaining/}}



%
% Footnote na lateral estah definido aqui
%
\renewcommand*{\raggedleftmarginnote}{}
\renewcommand*{\raggedrightmarginnote}{\flushleft}
\newcounter{myfootnote}
\renewcommand{\footnote}[2][0]{\refstepcounter{footnote}{\color{red}\mbox{\textsuperscript{$\thefootnote$}}}{\marginnote{ \RaggedRight\footnotesize\singlespacing {\color{red}\mbox{\textsuperscript{$\thefootnote$}}}#2}[#1\baselineskip]}}




%
% Redefinindo o titulo
%
\makeatletter
\def\@makechapterhead#1{%
  {\parindent \z@ \raggedright \normalfont
    \ifnum \c@secnumdepth >\m@ne
    	\begin{flushright}
        {\tikz 
        	\node [rectangle, minimum size=1.5cm, thick, draw=Black, top color=Black, bottom color=Black] (captit) at (0,0) {\color{white} \ \ \large\bfseries \@chapapp\space \thechapter \ \ };
        }
    	\end{flushright}
    \fi
    \vspace{-0.2cm}
    \interlinepenalty\@M
    \centering
    \huge \color{blue} \bfseries #1\par\nobreak
    \vskip 40\p@
  }}
\def\@makeschapterhead#1{%
  {\parindent \z@ \raggedright \normalfont
    \interlinepenalty\@M
    \huge \color{Black} \bfseries #1\par\nobreak
    \vskip 40\p@
  }}
\makeatother



%
% Primeiras definicoes do fancyheader
%
\fancyhead{}
\fancyfoot{}
\renewcommand{\headrulewidth}{1pt}
\renewcommand{\familydefault}{\sfdefault}
\pagestyle{fancy}


%
% Primeira chamada do tikz
%
\usetikzlibrary{arrows,shapes,automata,snakes,backgrounds,trees}
\tikzstyle{every picture}+=[remember picture]
